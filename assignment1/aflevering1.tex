\documentclass[10pt,a4paper,danish]{article}
\usepackage[danish]{babel}
\usepackage[utf8]{inputenc}
\usepackage{amsmath}
\usepackage{amssymb}
\usepackage{listings}
\usepackage{fancyhdr}
\usepackage[hidelinks]{hyperref}
\usepackage{booktabs}
\usepackage{graphicx}
\usepackage{xfrac}
\usepackage[dot, autosize, outputdir="dotgraphs/"]{dot2texi}
\usepackage{tikz}
\usepackage{ulem}
\usetikzlibrary{shapes}

\pagestyle{fancy}
\fancyhead{}
\fancyfoot{}
\rhead{\today}
\rfoot{\thepage}
\setlength\parskip{1em}
\setlength\parindent{1em}

%% Titel og forfatter
\title{ITS - aflevering 1}
\author{Søren Pilgård, 190689, vpb984\\
René Løwe Jacobsen, 070192, vlx198}

%% Start dokumentet
\begin{document}

%% Vis titel
\maketitle
\newpage

%% Vis indholdsfortegnelse
\tableofcontents
\newpage

\section{}
% % spray() (kaldes i init)
% kalder garbage collectoren
% fylder \_sarrays med 2.808.685 * 11 = tal
% fylder \_sleaks med 2.000.000 Leaks
% fanger OutOfMemoryError

% på en 32bits arkitektur optager det i hvertfald {(2808685 * 11 + 2000000) * 4
% bytes = 131.6MB
% (ikke medregnet pladsen selve Leaks objekterne fylder
% Men da disse blot indeholder 4 referencer og et tal er de ikke specielt store.

% spray gør dermed ikke rigtigt noget andet end at brænde CPUtid af.

% Når spray har fyldt ram, så kan programmet regne sig frem til, hvor system biblioteket ligger.
% Samt regne sig frem til, hvor SecurityManager klassen ligger og derefter overskrive den.
% Når SecurityManager er overskrevet i RAM har de fulde rettigheder og kan derfor lægge en mere inficeret fil ned på disken og køre den.


% 131582140
% Leaks indeholder en reference

\section{}

\section{Sideværts bevægelser}

\begin{verbatim}
ipconfig /all
\end{verbatim}
Oplister TCP/IP konfigurationen for alle netværksinterfaces.
På den måde kan hackeren se maskinens ip-adresser har på netværkene.

\begin{verbatim}
tasklist -v
\end{verbatim}
Oplister alle de programmer og services der eksekverer. v flaget giver ekstra
mange informationer.

\begin{verbatim}
netstat -ano
\end{verbatim}
Fortæller hvilke aktive TCP og UDP forbindelser der er, samt hvilke porte der
lyttes på. \texttt{-o} flaget giver proces ID'et med på programmerne bag.
Sammen holdt med informationerne fra tasklist kan man dermed se hvilke
programmer der snakker over hvilke forbindelser og porte på maskinen.

Her efter snakker hackeren med en række services

\begin{verbatim}
net user
\end{verbatim}
Lister alle brugerne på maskinen.

\begin{verbatim}
net start
\end{verbatim}
Lister alle services der kører på systemet.

\begin{verbatim}
net share
\end{verbatim}
Viser information om alle de ressourcer på systemet der er delt på netværket.

\begin{verbatim}
net localgroup administrators
\end{verbatim}
Viser alle de brugere der er med i brugergruppen af administratorer.

\begin{verbatim}
net group "domain users" /domain
\end{verbatim}
Viser hvilke brugere der findes i gruppen \textit{domain users} på hele domænet.


Det ses at \textit{enum.bat.txt} er et script der samler information om
systemets programmer, trafik samt brugere og gemmer det i en tekstfil \texttt{C:\\WINNT\\Debug\\1.txt} som
hackeren efterfølgende kan indsamle.

Hackeren kunne have indsat en kommando der slettede scriptet når det blev kørt
så det blev sværere for administratorene at finde ud af hvad hackeren foretog
sig på systemet.

Derudover kunne hackeren have lavet en portscanning af maskinerne på det interne
netværk for at finde hvilke processer der kører og lytter internt.
Disse programmer kan være kodet med et lavere sikkerheds niveau da de ikke er
forventet at have adgang til nettet udadtil.


\section{Lokal privilegiums forøgelse}

Der er to fejl:

For det første er programmet sårbart overfor ``format string'' angreb i kaldet
\texttt{snprintf}.

I kaldet kopieres \texttt{user\_input} over i \texttt{phone}, argumentet
\texttt{sizeof(phone} sikrer at der ikke placeres data udenfor det allokerede
array.

Men når man benytter sig af snprintf fortolkes strengen og eventuelle special
tegn evalueres og erstattes. Dermed kan en ondsindet bruger indsætte kommandoer
i strengen der kan få programmet til at crashe eller endnuværre udlevere data
fra hukommelsen ved at bruge \texttt{\%s} og \texttt{\%x}.

Man kan rette denne fejl ved at rette linjen til:
\begin{verbatim}
snprintf(phone, sizeof(phone), "%s", user_input);
\end{verbatim}
Dermed bliver \texttt{user\_input} indsat som en streng som et argument til
selve format strengen.


Den anden fejl er at man bruger \texttt{gets} til at modtage data.
\texttt{gets} læser data og gemmer det i \texttt{user\_input} indtil den møder et linjeskift eller et end-of-file
tegn.
Problemet er at vi ingen kontrol har med hvor mange tegn der rent faktisk bliver
lagt ned i \texttt{user\_input}, dette er et problem da \texttt{user\_input} er
et array med allokeret plads til 100 chars.

Hvis en ondsindet bruger sender mere end 100 tegn begynder man dermed at skrive
data i hukommelsen uden for arrayet.
I bedste fald rammer man noget skrivebeskyttet hukommelse, i værste tilader det
hackeren at overskriver returaddresen der ligger på stakken hvormed han kan
styre hvorfra programmet skal eksekvere. På denne måde kan hackeren få adgang
til at eksekvere arbitrær kode på maskinen og give ham shell-adgang.

Da programmet kører som root via setuid, vil han have root adgang på hele
maskinen.

\section{Bagmændene}
Nitro angrebene var en række angreb på forskellige internationale kemi firmaer,
deriblandt en række danske.

Nitro angrebene virkede ved at der blev sendt en mail til nogle af virksomhedens
medarbejdere. Mailen havde en vedhæftet fil der indeholdt trojaneren PoisonIvy
(en ``RAT'') der gav bagmændene adgang til
den inficerede maskine. Filen lignede tilsyneladende en tekst fil men bestod i
virkeligheden af en eksekverbar fil indeholdende trojaneren der efterfølgende
forsøgte at ringe tilbage til en ``Command \& Control'' server der gav hackeren
adgang til fjernstyring.
Herfra lykkedes det dem at opsnappe informationer om
brugere og passwords der gav dem mulighed for at hente den ønskede information
fra serverne.

Ifølge Symantec der opdagede angrebene, var formålet formententligt industri
spionage. Symantec har ligeledes identificeret en af bagmændene fra Kina.

Symantecs rapport\cite{nitro-article} beskriver ligeledes i deres rapport at der i kølvandet har været en
række andre lignende angreb fra tilsvarende grupper. Det er dermed svært at
afgøre om bagmanden bag angrebet på BIOchem er dem fra nitro angrebene.
Selvom den specifikke tekniske tilgang og filerne brugt til at starte angrebet
er forskelligt er den generelle metode dog ens.

Det kan dermed, med de nuværende informationer, ikke afgøres om det konkret er
de samme personer der står bag angrebene, men man kan godt formode at motiverne
er de samme og at formålet har været industri spionage mod BIOchem.


Som med mange andre angreb, har angrebsvektoren været igennem mails med
medarbejderne.
En oplagt tilgang til at begrænse fremtidige angreb er at uddanne brugerne til
ikke at åbne filer fra ubetroede kilder.
Eventuelt kan man sætte et filter op i mailserveren der helt filtrere sådanne
mails fra, f.eks. ved at skanne indholdet af filerne igennem eller blokere for
vedhæftninger fra mails udefra.



\section{Litteraturliste}

\begin{thebibliography}{9}
\bibitem{marchewka} Jack T. Marchewka, \textit{\LaTeX: Information Technology
    Project Management}.  John Wiley \& Sons, Inc.  3nd Edition, 2010.

\bibitem{kontrakt} \textit{e-TL Kontrakt} 26. juni 2006

\bibitem{windows} Microsoft, Command-line reference A-Z,
  \url{http://www.microsoft.com/resources/documentation/windows/xp/all/proddocs/en-us/ntcmds.mspx?mfr=true}.

\bibitem{format} \textit{Format String Attacks},
  \url{http://en.wikipedia.org/wiki/Format_string_attack}

\bibitem{nitro-article} Symantec \textit{The Nitro Attacks,S
tealing Secrets from the Chemical Industry} \url{http://www.symantec.com/content/en/us/enterprise/media/security_response/whitepapers/the_nitro_attacks.pdf}

\end{thebibliography}

\end{document}
