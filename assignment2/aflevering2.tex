\documentclass[10pt,a4paper,danish]{article}
\usepackage[danish]{babel}
\usepackage[utf8]{inputenc}
\usepackage{amsmath}
\usepackage{amssymb}
\usepackage{listings}
\usepackage{fancyhdr}
\usepackage[hidelinks]{hyperref}
\usepackage{booktabs}
\usepackage{graphicx}
\usepackage{xfrac}
\usepackage[dot, autosize, outputdir="dotgraphs/"]{dot2texi}
\usepackage{tikz}
\usepackage{ulem}
\usepackage{lscape}
\usetikzlibrary{shapes}

\pagestyle{fancy}
\fancyhead{}
\fancyfoot{}
\rhead{\today}
\rfoot{\thepage}
\setlength\parskip{1em}
\setlength\parindent{1em}

%% Titel og forfatter
\title{ITS - aflevering 2}
\author{Søren Pilgård, 190689, vpb984\\
  René Løwe Jacobsen, 070192, vlx198}

%% Start dokumentet
\begin{document}

%% Vis titel
\maketitle
\newpage

%% Vis indholdsfortegnelse
\tableofcontents
\newpage

\section{Netværks opdeling}
Den ekstra firewall er en god ting. Den giver en bedre opdeling af netværket, så
en kompromiteret computer fra workstation netværket ikke har adgang til de
interne servere, udover de ting, der er specificeret i opgaven.
Det giver den fordel, at det er sværere at cracke de interne servere,
selvom en cracker har adgang til en computer på workstation netværket.

Vi er kommet frem til følgende firewallregler:

\begin{landscape}
\begin{verbatim}
#!/bin/bash
#BIOchem's new firewall

#IP-addresses
fw_eth1="10.10.1.1"
fw_eth0="10.10.2.1"
workstation_net="10.10.1.0/24"
server_net="10.10.2.0/24"
mail_srv="130.255.254.17"
cluster="130.255.254.12"
internal_dns="10.10.2.25"
internal_ftp="10.10.2.33"
intraweb="10.10.2.7"
adminserver="10.10.2.25"

#Specific rules
iptables -A --state ESTABLISHED,RELATED -j ACCEPT # Allow connections already established
iptables -A -d mail_srv -s workstation_net -p tcp --dport 25 -j ACCEPT --state NEW # SMTP
iptables -A -d mail_srv -s workstation_net -p tcp,udp --dport 26 -j ACCEPT --state NEW # SMTP
iptables -A -d mail_srv -s workstation_net -p tcp--dport 995 -j ACCEPT --state NEW # SSL-POP
iptables -A -d mail_srv -s workstation_net -p tcp --dport 465 -j ACCEPT --state NEW # SSMTP
iptables -A -d intraweb -s workstation_net -p tcp --dport 80 -j ACCEPT --state NEW # HTTP
iptables -A -d intraweb -s workstation_net -p tcp --dport 443 -j ACCEPT --state NEW # HTTPS
iptables -A -d internal_ftp -s workstation_net -p tcp,udp --dport 20 -j ACCEPT --state NEW # FTP data
iptables -A -d internal_ftp -s workstation_net -p tcp --dport 21 -j ACCEPT --state NEW # FTP control
iptables -A -d cluster -s workstation_net -p tcp,udp --dport 22 -j ACCEPT --state NEW # SSH
iptables -A -d workstation_net -s adminserver -p icmp -j ACCEPT --state NEW # ping fra admin
iptables -A -d server_net --dport 80 -j DROP --state NEW # drop HTTP to servernet
iptables -A -d server_net --dport 443 -j DROP --state NEW # drop HTTPS to servernet
iptables -A -d internal_dns -s workstation_net --dport 53 -j ACCEPT --state NEW
iptables -A -s workstation_net --dport 80 -j ACCEPT # HTTP to internet
iptables -A -s workstation_net --dport 443 -j ACCEPT # HTTPS to internet

#Default policies
iptables -A -s workstation_net -j DROP
iptables -A -d workstation_net -j DROP

#Save the rules
# Her er det antaget, at det kører på et Debian baseret system
iptables-save > /etc/iptables/rules.v4 # for IPv4
iptables-save > /etc/iptables/rules.v6 # for IPv6
\end{verbatim}
\end{landscape}

\section{Kompromittering af klyngens indgang}
Udfra logfilen kan vi se, at angriberen har åbnet to SSH forbindelser ad gangen.
Her forsøger han at gætte løsenet til root brugeren på serveren. Hvis det ikke
lykkes på nogen af de to forbindelser efter 6 gange, så åbner han to nye igen og
prøver på dem.
Efter utallige forsøg får angriberen adgang til serveren på root brugeren, så
reelt set er alle brugere kompromiterede på serveren, da root har adgang til
alt.

Hvordan kunne man forhindre dette?
For det første bør man konfigurere sshd bedre.
\texttt{PermitRootLogin} bør sættes som no, så man ikke kan logge ind som root
bruger, maksimalt til en bruger på maskinen med sudo adgang.

\texttt{PasswordAuthentication} skal sættes til no.
Brugeres passwords er stortset altid elendige rent sikkerhedsmæsigt. I stedet
bør brugerens offentlige sshnøgle tilføjes til listen af authentikerede nøgler.
Derved kan man kun logge ind såfremt ens (lange) nøgle ligger på systemet, denne
tager meget lang tid at bruteforce.

\texttt{LoginGraceTime} bør sættes til en relativt lav værdi. Denne indstilling
fortæller hvor lang tid der maksimalt må gå fra en bruger forbinder til der
logges ind.

Man kan derudover rykke den port som sshd lytter på til en anden end port 22.
Det er ``Security through obscurity'', en dedikeret angriber vil blot lave et
port scan og dermed finde frem til sshd porten. Men der er utroligt mange angreb
der foregår via scripts der automatisk angriber ssh daemoner på port 22.
Disse angreb løber bare hen over så mange maskiner som muligt og angriber port
22 i et forsøg på f.eks. at opbygge et botnet.

Man bør derudover overveje hvilke IP adresser der skal have adgang til at
forbinde til maskinen. Dette kan gøres gennem firewallen der kan konfigureres
til kun at tillade trafik til serveren inde fra netværket.
For at give forskerne adgang hjemmefra kan man tillade deres hjemmeIP'er,
alternativt kan der benyttes VPN til at forbinde forskernes hjemmenetværk til
BIOchems.

Derudover kan man konfigurere firewallen til at droppe forbindelser der for
mange gange i træk prøver at forbinde til sshd porten

f.eks. gør følgende opsætning\cite{portdrop} til IPtable at hvis der forbindes mere end 5 gange
til porten inden for 60 sekunder, da bliver pakkerne droppet.
\begin{verbatim}
#!/bin/bash
inet_if=eth1
ssh_port=22
$IPT -I INPUT -p tcp --dport ${ssh_port} -i ${inet_if}
                 -m state --state NEW -m recent  --set

$IPT -I INPUT -p tcp --dport ${ssh_port} -i ${inet_if}
               -m state --state NEW -m recent --update
                     --seconds 60 --hitcount 5 -j DROP
\end{verbatim}


THC Hydra er et program til automatisk at angribe en maskine, den får en ip og en
port og så forsøger den at logge ind. Den bruger et dictionary af ord som den
forsøger at angribe med.
På den måde minder den om en avanceret udgave af de før omtalte scripts til
automatiske angreb.

For at gøre det sværre for BIOchem at finde ud af hvordan angrebet foregik,
kunne man som root have slettet/redigeret loggen så man ikke kan se hvad der var
sket.


\section{Brud på løsener}

Vi har fået adgang til en hashet fil med brugernavne og password hashes.
Vi ser at filen har formatet \textit{brugernavn} \texttt{:} \textit{tal}
\texttt{:} \textit{hash 1} \texttt{:} \textit{hash 2} \texttt{:::}

Der er 32 tegn i hvert hash med tegnene \textit{0-9+A-F}, det tyder på at det er
hexadecimale tal. Vi har dermed:
\[16^{32} = 2^{128}\]
Det vil sige at vi i værste tilfælde har et hash på 128 bits, tilsvarende md5.

I praksis viser det sig dog at selve sikkerheden er lavere.

SAM filer viser sig at bestå af følgende format\cite{cracking}:\\
\textit{brugernavn} \texttt{:} \textit{Security ID}
\texttt{:} \textit{LM hash} \texttt{:} \textit{NTLM hash} \texttt{:::}
\begin{itemize}
\item \textit{Security ID}'et er et unikt tal per bruger.
\item \textit{LM hash} er et hash givet ved LanMan hashing functionen
\item \textit{NTLM hash} er et hash givet ved NTLanMan hashing functionen
\end{itemize}

LM hash er en forældet og brudt hashing funktion som normalt er deaktiveret
hvorved feltet burde indeholde teksten ``\texttt{NO
  PASSWORD*********************}''.

Når dette ikke er tilfældet må LM hashing være slået til manuelt.
Vi kan dermed bryde hashet, for det første ved vi at LM kun kigger på de første
14 tegn i løsnet, hvis vi antager at der kun bruges ascii tegn vil der være 95
printbare tegn hvilket giver \(95^{14} \approx 2^{92}\) bits.
Desværre omdannes alle tegn til deres greltegns udgave hvilket giver \(69^{14}
\approx 2^{86}\)

Derudover bliver løsnet delt op i to grupper af 7 tegn hashet individuelt,
for hver gruppe skal man bryde \(69^{7} \approx \ 2^{43}\).
Hvis løsnet er på 7 tegn eller mindre er dette alt der skal brydes, hvis det er
længere skal man bryde \(2 * 69^{7} \approx 2^{44}\) bits.

At bryde 44 bits tager ikke lang tid.
Specielt fordi man ikke bruger et ``salt'', dermed kan man opbygge en tabel én
gang og så blot søge over denne, et såkaldt ``rainbowtable''.


NTLM er heller ikke specielt sikker, Microsoft har selv meldt ud at man ikke bør
bruge den. NTLM benytter som LM heller ikke et hash og kan dermed brydes med et
rainbowtable.

Det er trivielt at finde tjenester\cite{cracker1} \cite{cracker2} på nettet der, givet et LM eller NTLM hash, finder
klarteksten.

\begin{table}[h!]
  \centering
  \begin{tabular}{ll}
    \textit{Brugernavn} & \textit{Løsen} \\
    \hline
    Administrator & \texttt{@P0WNDK3*\{1!\}}\\
    Anni & \texttt{Anni}\\
    Lis & \texttt{nusser}\\
    Peter & \texttt{bmwX5}\\
    Support & \texttt{sommer2010}
  \end{tabular}
\end{table}

For bedre at sikre løsenerne bør man slå LM hashing fra. Ligeledes bør man sikre
sig at brugerne vælger løsener der er tilpas lange og sofistikerede (små, store
og specialtegn).

SAM er tilsyneladende generelt ikke en specielt sikker løsning. BIOchem bør i stedet se på om
man ikke kan benytte \textit{Active Directory} i stedet.

Passwords bør generelt hashes med en solid passwordhashing algoritme der er
tilpas langsom og der bør benyttes et kryptografisk ``salt''.

\section{Litteraturliste}

\begin{thebibliography}{9}

\bibitem{cracking}
  \url{http://www.onlinehashcrack.com/how_to_crack_windows_passwords.php}, 21.
  maj 2014.

\bibitem{donotuse} Microsoft Developer Network,
  \url{http://msdn.microsoft.com/en-us/library/cc236715.aspx}

\bibitem{cracker1} \url{http://www.hashkiller.co.uk/ntlm-decrypter.aspx}
\bibitem{cracker2} \url{http://rainbowtables.it64.com/}

\bibitem{portdrop} \url{http://www.cyberciti.biz/tips/linux-unix-bsd-openssh-server-best-practices.html}
\end{thebibliography}


\end{document}
