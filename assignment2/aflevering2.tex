\documentclass[10pt,a4paper,danish]{article}
\usepackage[danish]{babel}
\usepackage[utf8]{inputenc}
\usepackage{amsmath}
\usepackage{amssymb}
\usepackage{listings}
\usepackage{fancyhdr}
\usepackage[hidelinks]{hyperref}
\usepackage{booktabs}
\usepackage{graphicx}
\usepackage{xfrac}
\usepackage[dot, autosize, outputdir="dotgraphs/"]{dot2texi}
\usepackage{tikz}
\usepackage{ulem}
\usetikzlibrary{shapes}

\pagestyle{fancy}
\fancyhead{}
\fancyfoot{}
\rhead{\today}
\rfoot{\thepage}
\setlength\parskip{1em}
\setlength\parindent{1em}

%% Titel og forfatter
\title{ITS - aflevering 2}
\author{Søren Pilgård, 190689, vpb984\\
  René Løwe Jacobsen, 070192, vlx198}

%% Start dokumentet
\begin{document}

%% Vis titel
\maketitle
\newpage

%% Vis indholdsfortegnelse
\tableofcontents
\newpage

\section{Netværks opdeling}
Der er fler

- opdelingen er god
    - giver mindre adgang fra workstations, hvilket gør, at der er mindre
      sandsynlighed for, at de kan angribes, hvis en workstation bliver
      kompromiteret.

\section{Kompromittering af klyngens indgang}

\section{Brud på løsener}

Vi har fået adgang til en hashet fil med brugernavne og password hashes.
Vi ser at filen har formatet \textit{brugernavn} \texttt{:} \textit{tal}
\texttt{:} \textit{hash 1} \texttt{:} \textit{hash 2} \texttt{:::}

Der er 32 tegn i hvert hash med tegnene \textit{0-9+A-F}, det tyder på at det er
hexadecimale tal. Vi har dermed:
\[16^{32} = 2^{128}\]
Det vil sige at vi i værste tilfælde har et hash på 128 bits, tilsvarende md5.

I praksis viser det sig dog at selve sikkerheden er lavere.

SAM filer viser sig at bestå af følgende format\cite{cracking}:\\
\textit{brugernavn} \texttt{:} \textit{Security ID}
\texttt{:} \textit{LM hash} \texttt{:} \textit{NTLM hash} \texttt{:::}
\begin{itemize}
\item \textit{Security ID}'et er et unikt tal per bruger.
\item \textit{LM hash} er et hash givet ved LanMan hashing functionen
\item \textit{NTLM hash} er et hash givet ved NTLanMan hashing functionen
\end{itemize}

LM hash er en forældet og brudt hashing funktion som normalt er deaktiveret
hvorved feltet burde indeholde teksten ``\texttt{NO
  PASSWORD*********************}''.

Når dette ikke er tilfældet må LM hashing være slået til manuelt.
Vi kan dermed bryde hashet, for det første ved vi at LM kun kigger på de første
14 tegn i løsnet, hvis vi antager at der kun bruges ascii tegn vil der være 95
printbare tegn hvilket giver \(95^{14} \approx 2^{92}\) bits.
Desværre omdannes alle tegn til deres greltegns udgave hvilket giver \(69^{14}
\approx 2^{86}\)

Derudover bliver løsnet delt op i to grupper af 7 tegn hashet individuelt,
for hver gruppe skal man bryde \(69^{7} \approx \ 2^{43}\).
Hvis løsnet er på 7 tegn eller mindre er dette alt der skal brydes, hvis det er
længere skal man bryde \(2 * 69^{7} \approx 2^{44}\) bits.

At bryde 44 bits tager ikke lang tid.
Specielt fordi man ikke bruger et ``salt'', dermed kan man opbygge en tabel én
gang og så blot søge over denne, et såkaldt ``rainbowtable''.


NTLM er heller ikke specielt sikker, Microsoft har selv meldt ud at man ikke bør
bruge den. NTLM benytter som LM heller ikke et hash og kan dermed brydes med et
rainbowtable.

Det er trivielt at finde tjenester\cite{cracker1} \cite{cracker2} på nettet der, givet et LM eller NTLM hash, finder
klarteksten.

\begin{table}[h!]
  \centering
  \begin{tabular}{ll}
    \textit{Brugernavn} & \textit{Løsen} \\
    \hline
    Administrator & \texttt{@P0WNDK3*\{1!\}}\\
    Anni & \texttt{Anni}\\
    Lis & \texttt{nusser}\\
    Peter & \texttt{bmwX5}\\
    Support & \texttt{sommer2010}
  \end{tabular}
\end{table}

For bedre at sikre løsenerne bør man slå LM hashing fra. Ligeledes bør man sikre
sig at brugerne vælger løsener der er tilpas lange og sofistikerede (små, store
og specialtegn).

SAM er tilsyneladende generelt ikke en specielt sikker løsning. BIOchem bør i stedet se på om
man ikke kan benytte \textit{Active Directory} i stedet.

Passwords bør generelt hashes med en solid passwordhashing algoritme der er
tilpas langsom og der bør benyttes et kryptografisk ``salt''.

\section{Litteraturliste}

\begin{thebibliography}{9}

\bibitem{cracking}
  \url{http://www.onlinehashcrack.com/how_to_crack_windows_passwords.php}, 21.
  maj 2014.

\bibitem{donotuse} Microsoft Developer Network,
  \url{http://msdn.microsoft.com/en-us/library/cc236715.aspx}

\bibitem{cracker1} \url{http://www.hashkiller.co.uk/ntlm-decrypter.aspx}
\bibitem{cracker2} \url{http://rainbowtables.it64.com/}

\end{thebibliography}


\end{document}
