\documentclass[10pt,a4paper,danish]{article}
\usepackage[danish]{babel}
\usepackage[utf8]{inputenc}
\usepackage{amsmath}
\usepackage{amssymb}
\usepackage{listings}
\usepackage{fancyhdr}
\usepackage[hidelinks]{hyperref}
\usepackage{booktabs}
\usepackage{graphicx}
\usepackage{xfrac}
\usepackage[dot, autosize, outputdir="dotgraphs/"]{dot2texi}
\usepackage{tikz}
\usepackage{ulem}
\usetikzlibrary{shapes}

\pagestyle{fancy}
\fancyhead{}
\fancyfoot{}
\rhead{\today}
\rfoot{\thepage}
\setlength\parskip{1em}
\setlength\parindent{1em}

%% Titel og forfatter
\title{ITS - aflevering 2}
\author{Søren Pilgård, 190689, vpb984\\
  René Løwe Jacobsen, 070192, vlx198}

%% Start dokumentet
\begin{document}

%% Vis titel
\maketitle
\newpage

%% Vis indholdsfortegnelse
\tableofcontents
\newpage

\section{Netværks opdeling}
Der er fler

- opdelingen er god
    - giver mindre adgang fra workstations, hvilket gør, at der er mindre
      sandsynlighed for, at de kan angribes, hvis en workstation bliver
      kompromiteret.

\section{Kompromittering af klyngens indgang}

\section{Brud på løsener}


\end{document}
