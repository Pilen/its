\documentclass[10pt,a4paper,danish]{article}
\usepackage[danish]{babel}
\usepackage[utf8]{inputenc}
\usepackage{amsmath}
\usepackage{amssymb}
\usepackage{listings}
\usepackage{fancyhdr}
\usepackage[hidelinks]{hyperref}
\usepackage{booktabs}
\usepackage{graphicx}
\usepackage{xfrac}
\usepackage[dot, autosize, outputdir="dotgraphs/"]{dot2texi}
\usepackage{tikz}
\usepackage{ulem}
\usepackage{lscape}
\usetikzlibrary{shapes}

\pagestyle{fancy}
\fancyhead{}
\fancyfoot{}
\rhead{\today}
\rfoot{\thepage}
\setlength\parskip{1em}
\setlength\parindent{1em}

%% Titel og forfatter
\title{ITS - aflevering 2}
\author{Søren Pilgård, 190689, vpb984\\
  René Løwe Jacobsen, 070192, vlx198}

%% Start dokumentet
\begin{document}

%% Vis titel
\maketitle
\newpage

%% Vis indholdsfortegnelse
\tableofcontents
\newpage

\section{Et online presserum}
Det problem, der springer i øjnene er, at koden giver mulighed for at åbne en
hvilken som helst fil på serveren, som den bruger webserveren kører på kan læse.
Dette kan gøres ved at ændre på GET-parametret pub og bruge ../ for at gå tilbage
mod roden på disken.

Et fix til dette problem kunne være at putte PHP applikationen i et jail, så det
ikke er muligt at komme udfra rodmappen for applikationen.
Man kunne også gøre sådan, at man i stedet for lavede linksne til filerne
direkte, så de ikke blev loaded af PHP.


\section{2}

\section{3}

\section{4}

\section{Litteraturliste}

\begin{thebibliography}{9}

\bibitem{cracking}
  \url{http://www.onlinehashcrack.com/how_to_crack_windows_passwords.php}, 21.
  maj 2014.

\bibitem{donotuse} Microsoft Developer Network,
  \url{http://msdn.microsoft.com/en-us/library/cc236715.aspx}

\bibitem{cracker1} \url{http://www.hashkiller.co.uk/ntlm-decrypter.aspx}
\bibitem{cracker2} \url{http://rainbowtables.it64.com/}

\end{thebibliography}


\end{document}
