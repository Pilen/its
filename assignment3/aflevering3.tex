\documentclass[10pt,a4paper,danish]{article}
\usepackage[danish]{babel}
\usepackage[utf8]{inputenc}
\usepackage{amsmath}
\usepackage{amssymb}
\usepackage{listings}
\usepackage{fancyhdr}
\usepackage[hidelinks]{hyperref}
\usepackage{booktabs}
\usepackage{graphicx}
\usepackage{xfrac}
\usepackage[dot, autosize, outputdir="dotgraphs/"]{dot2texi}
\usepackage{tikz}
\usepackage{ulem}
\usepackage{lscape}
\usetikzlibrary{shapes}

\pagestyle{fancy}
\fancyhead{}
\fancyfoot{}
\rhead{\today}
\rfoot{\thepage}
\setlength\parskip{1em}
\setlength\parindent{1em}

%% Titel og forfatter
\title{ITS - aflevering 2}
\author{Søren Pilgård, 190689, vpb984\\
  René Løwe Jacobsen, 070192, vlx198}

%% Start dokumentet
\begin{document}

%% Vis titel
\maketitle
\newpage

%% Vis indholdsfortegnelse
\tableofcontents
\newpage

\section{Et online presserum}
Det problem, der springer i øjnene er, at koden giver mulighed for at åbne en
hvilken som helst fil på serveren, som den bruger webserveren kører på kan læse.
Dette kan gøres ved at ændre på GET-parametret pub og bruge ../ for at gå tilbage
mod roden på disken.

Et fix til dette problem kunne være at putte PHP applikationen i et jail, så det
ikke er muligt at komme udfra rodmappen for applikationen.
Man kunne også gøre sådan, at man i stedet for lavede linksne til filerne
direkte, så de ikke blev loaded af PHP.

\section{Adskillelse af opgaver}
Til at starte med vil vi lige sige, at vi har antaget, at det løsen vi får
ind er i hexadecimal form, hvilket også passer på løsenet vi får i opgaven.

For at løse problemet generelt har vi valgt at bruge Shamir's threshold secret
sharing. Her har vi valgt at gøre det med en ret linje, da vi skal kunne komme
tilbage til adgangskoden vha. 2 nøgler.

Implementationen er skrevet i python og ser således ud:
\begin{verbatim}
"""
These algorithms are based on Shamir's secret sharing.
As we only need to keys to get the original, we'll be using a line.

A key will look like this [x, y]
"""
import random

def generateKeys(key, keysWanted):
    a = random.randint(1,1000000000000)
    b = int(key, 16)

    keys = []
    for i in range(0, keysWanted):
        x = random.randint(1, 1000000000000)
        y = str(a*x+b)
        key = "0" * (len(y) - len(str(x))) + str(x) + str(y)
        keys.append(key)

    return keys

def getOriginalKey(key1, key2):
    x1 = int(key1[:len(key1)/2])
    x2 = int(key2[:len(key2)/2])
    y1 = int(key1[len(key1)/2:])
    y2 = int(key2[len(key2)/2:])
    a = (y2 - y1) / (x2 - x1)
    b = y1 - x1 * a
    return format(b, 'x')
\end{verbatim}

Kaldes \texttt{generateKeys} med argumenterne 1af84eb2c98 og 3, så kunne
outputtet se således ud, \texttt{['00000001867731863824306470', '00000008381061900313283796', '00000007970061898010779596']}.
Disse nøgler kan så sendes ud til hhv. BIOchem’s leder af kommunikationsafdeling,
souschefen and højeststående pressemedarbejder.

\subsection{Hvis en nøgle bliver kompromiteret}
Hvis en nøgle bliver kompromiteret, så kan der genereres nye nøgler ved at kalde
\texttt{generateKeys} med det samme input igen, hvilket burde give nye nøgler,
der kan sendes ud til brugerne.
Det er dog muligt, at der ved denne kørsel af \texttt{generateKeys} bliver lavet
en linje med samme hældning, da hældningen for den gamle ikke er kendt og bliver
"tilfældigt" genereret.
Det kan være et problem, hvis en hacker har en keylogger på en af nøglehavernes
maskiner og derved kan opsnappe en nøgle mere, så han har fuld adgang.
For at undgå dette kan de køre \texttt{getOriginalKey} en gang med en af de
gamle nøgler og en af de nye og se, om de evt. giver den rigtige. Hvis resultatet
bliver det rigtige, så kan \texttt{generateKeys} køres igen.

\section{3}

\section{4}

\section{Litteraturliste}

\begin{thebibliography}{9}

\bibitem{cracking}
  \url{http://www.onlinehashcrack.com/how_to_crack_windows_passwords.php}, 21.
  maj 2014.

\bibitem{donotuse} Microsoft Developer Network,
  \url{http://msdn.microsoft.com/en-us/library/cc236715.aspx}

\bibitem{cracker1} \url{http://www.hashkiller.co.uk/ntlm-decrypter.aspx}
\bibitem{cracker2} \url{http://rainbowtables.it64.com/}

\end{thebibliography}


\end{document}
